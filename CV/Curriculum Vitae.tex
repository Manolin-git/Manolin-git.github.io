\documentclass[10pt,a4paper,sans,colorlinks,linkcolor=true]{moderncv}        % possible options include font size ('10pt', '11pt' and '12pt'), paper size ('a4paper', 'letterpaper', 'a5paper', 'legalpaper', 'executivepaper' and 'landscape') and font family ('sans' and 'roman')



\usepackage{titlesec}
% moderncv themes
\moderncvstyle{casual}                             % style options are 'casual' (default), 'classic', 'banking', 'oldstyle' and 'fancy'
\moderncvcolor{burgundy}                               % color options 'black', 'blue' (default), 'burgundy', 'green', 'grey', 'orange', 'purple' and 'red'
%\renewcommand{\familydefault}{\sfdefault}         % to set the default font; use '\sfdefault' for the default sans serif font, '\rmdefault' for the default roman one, or any tex font name
%\nopagenumbers{}                                  % uncomment to suppress automatic page numbering for CVs longer than one page

% character encoding
%\usepackage[utf8]{inputenc}                       % if you are not using xelatex ou lualatex, replace by the encoding you are using
%\usepackage{CJKutf8}                              % if you need to use CJK to typeset your resume in Chinese, Japanese or Korean



% adjust the page margins
\usepackage[scale=0.75, top=3cm, bottom=3cm]{geometry}
%\setlength{\hintscolumnwidth}{3cm}                % if you want to change the width of the column with the dates
%\setlength{\makecvheadnamewidth}{10cm}            % for the 'classic' style, if you want to force the width allocated to your name and avoid line breaks. be careful though, the length is normally calculated to avoid any overlap with your personal info; use this at your own typographical risks...
\setlength{\hintscolumnwidth}{3cm}

\geometry{
	left=12.8mm, % left margin
	textwidth=178mm, % main text block
	marginparsep=2.2mm, % gutter between main text block and margin notes
	marginparwidth=7.4mm % width of margin notes
}
\usepackage[symbol]{footmisc}
% personal data

\name{Himanshu}{Sahu}
\title{Curriculum Vit\ae{}}                               % optional, remove / comment the line if not wanted
%\address{street and number}{postcode city}{country}% optional, remove / comment the line if not wanted; the "postcode city" and "country" arguments can be omitted or provided empty
%\phone[mobile]{+1~(234)~567~890}                   % optional, remove / comment the line if not wanted; the optional "type" of the phone can be "mobile" (default), "fixed" or "fax"
%\phone[fixed]{+2~(345)~678~901}
%\phone[fax]{+3~(456)~789~012}


\email{himanshusah1@iisc.ac.in}                               % optional, remove / comment the line if not wanted
%\homepage{https://manolin-git.github.io//}                         % optional, remove / comment the line if not wanted
\social[linkedin]{himanshu-sahu-iisc}                        % optional, remove / comment the line if not wanted
%\social[xing]{john\_doe}                           % optional, remove / comment the line if not wanted
%\social[twitter]{jdoe}                             % optional, remove / comment the line if not wanted
\social[github]{Manolin-Git}                             % optional, remove / comment the line if not wanted
%\social[gitlab]{jdoe}                              % optional, remove / comment the line if not wanted
%\social[skype]{jdoe}                               % optional, remove / comment the line if not wanted
%\extrainfo{additional information}                 % optional, remove / comment the line if not wanted
%\photo[100pt][0pt]{picture}                       % optional, remove / comment the line if not wanted; '64pt' is the height the picture must be resized to, 0.4pt is the thickness of the frame around it (put it to 0pt for no frame) and 'picture' is the name of the picture file
\quote{We have art in order not to die of the truth. -- Nietzsche}                                 % optional, remove / comment the line if not wanted

% bibliography adjustements (only useful if you make citations in your resume, or print a list of publications using BibTeX)
%   to show numerical labels in the bibliography (default is to show no labels)
%\makeatletter\renewcommand*{\bibliographyitemlabel}{\@biblabel{\arabic{enumiv}}}\makeatother
\renewcommand*{\bibliographyitemlabel}{[\arabic{enumiv}]}
%   to redefine the bibliography heading string ("Publications")
%\renewcommand{\refname}{Articles}

% bibliography with mutiple entries
\usepackage{multibib}
\newcites{book,misc}{{Books},{Others}}
\usepackage{multicol} 
\definecolor{true}{HTML}{40916C}
\definecolor{false}{HTML}{9D0208}
\definecolor{darkgray}{HTML}{495057}
\usepackage{graphicx}
\definecolor{sagegreen}{HTML}{9C7C82}
\definecolor{bury}{HTML}{800020}


\makeatletter
\NewDocumentCommand{\mysubsection}{sm}{%
	\par\addvspace{2ex}%
	\phantomsection{}% reset the anchor for hyperrefs
	\addcontentsline{toc}{subsection}{#2}%
	{\strut\raggedleft\raisebox{\baseletterheight}{\color{bury!80!white}\hspace{1.4cm}\rule{0.5\hintscolumnwidth}{0.85ex}}\quad}{\strut\subsectionstyle{\textcolor{bury!80!white}{#2}}}%
	\par\nobreak\addvspace{.5ex}\@afterheading}% to avoid a pagebreak after the heading
\makeatother

%----------------------------------------------------------------------------------
%            content
%----------------------------------------------------------------------------------
\begin{document}
	%\begin{CJK*}{UTF8}{gbsn}                          % to typeset your resume in Chinese using CJK
	%-----       resume       ---------------------------------------------------------
	\makecvtitle
	\pagestyle{empty}
	
	\definecolor{links}{HTML}{AA4A44}
	\hypersetup{urlcolor=links}
	%\hypersetup{
		%		colorlinks=true,
		%		linkcolor=blue,
		%		filecolor=red,      
		%		urlcolor=false,citecolor=purple
		%	}
	
	
	
	
	\section{Personal Information}
	\cvitem{Nome e Cognome}{Himanshu Sahu}
	\cvitem{Gender}{Male}
	\cvitem{Date of Birth}{22-07-2002}
	\cvitem{Address}{\href{http://iap.iisc.ac.in/}{Department of Instrumental \& Applied Physics,}
		\href{https://iisc.ac.in/}{Indian Institute of Science},
		C.V. Raman Avenue, Bengaluru 560012, India}
	%\cvitem{E-mail}{himanshusah1@iisc.ac.in}
	\cvitem{Nationality}{Indian}
	
	\begin{center}
		\href{https://scholar.google.com/citations?user=o3SoQjUAAAAJ&hl=en}{Google Scholar} $\bullet$ \href{https://inspirehep.net/authors/2619981?ui-citation-summary=true&ui-exclude-self-citations=true}{iNSPIRE-HEP} $\bullet$ \href{https://orcid.org/0000-0002-9522-6592}{ORCiD} $\bullet$ \href{https://www.linkedin.com/in/himanshu-sahu-iisc/}{Linkedin} $\bullet$ \href{https://github.com/Manolin-git}{Github}
	\end{center}
	
		\section{Experience}
	\cventry{03/2024-Present}{Quantum Intern}{\textnormal{IBM Research Lab}}{Bangalore.}{}{}
	
	
	\section{Education}
	\color{gray}
	\cventry{Fall 2024}{PhD in Quantum Computation and Quantum Communication}{\textnormal{University College London}}{London, UK. \textcolor{blue}{This position is subject to changes.}}{}{$\bullet $ EPSRC Centre for Doctoral Training in Delivering Quantum Technologies}
	\color{black}
	\cventry{07/2021--Present}{Masters in Physics}{\textnormal{Indian Institute of Science}}{Bangalore.}{}{$\bullet$ CGPA : 9.10/10  (max. typically $\sim$ 9.4) $\bullet$ On track for Distinction in the Major}  % arguments 3 to 6 can be left empty
	\cventry{06/2018-06/2021}{Bachelor in Physics}{\textnormal{Banaras Hindu University}}{Varanasi.}{}{$\bullet$ CGPA : 9.15/10 $\bullet$ Passed in First Division with Distinction}
	
	\section{Master thesis}
	
	\cvitem{Title}{Quantum walk based simulations \& algorithms}
	\cvitem{Supervisor}{Prof. Subroto Mukerjee \& Prof. CM Chandrashekar}
	\cvitem{Description}{To be decided}
	
	
	\section{Research activity}
	\cvitem{Brief description}{I am a physicist broadly interested in the ideas at the intersection between condensed matter theory, quantum computing, and information theory.}
	
	\mysubsection{Research Interests}
	\cvitem{Main interests}{Quantum optics, Quantum information, Quantum computation, Quantum simulation, Quantum error correction, Quantum algorithms, Quantum sensing, Quantum many-body physics, Open quantum systems, Quantum communication, Quantum chaos.}
	%\cventry{Main results}{Neutrino oscillations in a quantum walk framework}{}{}{}{By viewing the position space of a quantum walk as an environment, I've developed a novel approach to simulate neutrino flavor change dynamics within an open quantum system framework. This reduces the resource required for simulating quantum walk which include position space.}
	
	\mysubsection{Research Experience}
	
	\cventry{05/2022-05/2023}{Neutrino oscillations in a quantum walk framework}{}{}{}{\textbf{PI:} Prof. CM Chandrashekar (Dept. of Instrumentation \& Applied Physics, Indian Institute of Sciences)\\ \textbf{Summary:} By viewing the position space of a quantum walk as an environment, I've developed a novel approach to simulate neutrino flavor change dynamics within an open quantum system framework. This reduces the resource required for simulating quantum walk which include position space. }
	
	%
	%$\bullet$ Developed a novel approach to simulate neutrino flavor change dynamics within an open quantum system framework $\bullet$ Derived generic recurrence relation to construct Kraus operators $\bullet$ Proposed generic scheme that reduces the resource required for simulating quantum walk.
	
	\cventry{05/2022-03/2023}{Krylov complexity in open systems}{}{}{}{\textbf{PI:} Dr. Aranya Bhattacharya (Centre for High Energy Physics, Indian Institute of Science)\\ \textbf{Team/Collaborators:} Pratik Nandy (Yukawa Institute for Theoretical Physics, Kyoto University), Pingal Pratyush Nath (Centre for High Energy Physics, Indian Institute of Science)  \\ 
		\textbf{Summary:} Our work extended the Krylov construction framework to dissipative open quantum systems coupled to a Markovian bath. This approach accommodates non-Hermitian effects due to the environment. Our investigation of the dissipative transverse-field Ising model and dissipative interacting XXZ chain reveals that initial Lanczos coefficients distinguish integrable and chaotic evolution during weak coupling. As dissipative effects intensify, higher Lanczos coefficients exhibit heightened fluctuations, culminating in similar late-time complexity saturation for both integrable and chaotic scenarios, casting doubt on the concept of late-time chaos.}
	
	
	\cventry{06/2023-08/2023}{Quantum circuit complexity of quantum walk}{}{}{}{\textbf{PI:} Dr. Kallol Sen (ICTP-South American Instutute of Fundamental Research) \& Dr. Aranya Bhattacharya (Institute of Physics, Jagiellonian University)\\
		\textbf{Team/Collaborators:} Dr. Ahmadullah Zahed (ICTP, Trieste)\\
		\textbf{Summary:} We studied circuit complexity of quantum walk. Notably, we unveil that the Nielson complexity during unitary evolution oscillates, centering around an average circuit depth. Additionally, we reveal that the complexity of the step-wise evolution operator exhibits cumulative and linear growth in relation to the number of steps taken. This observation, from a quantum circuit perspective, implies a sequential application of (approximately) constant-depth circuits, contributing to the attainment of the final state. We explicitly constructed the quantum circuit, and verified the observation. This study contribute to our understanding of relation between quantum complexity and circuit complexity.}
	
	
	
	
	\cventry{06/2023-01/2024}{Out-of-Time-Ordered Correlator's Growth Rate in a $\mathcal{PT}$-symmetric Chaotic System}{}{}{}{\textbf{PI:} Prof. Subroto Mukerjee (Department of Physics, Indian Institute of Science)\\
		\textbf{Team/Collaborators:} Kshitij Vijay Sharma (Department of Physics, Indian Institute of Science) \\ \textbf{Summary:} In ongoing work, we studied OTOC as well as complex level-spacing ratio as a diagnose of quantum chaos for the $\mathcal{PT}$-symmetric quantum kicked rotor -- a textbook driven chaotic system. The analysis based on complex level-spacing ratio shows that phase space consists of unbroken integrable and chaotic phases, and broken chaotic phase while broken integrable phase is absent. The OTOC shows exponential growth at early time in chaotic phase, as well as at late time in broken $\mathcal{PT}$-symmetric phase.} 
	
	\cventry{07/2023-10/2023}{Quantum search algorithm}{}{}{}{\textbf{PI:} Dr. Kallol Sen (ICTP-South American Instutute of Fundamental Research)\\ \textbf{Summary:} Building upon the quantum search algorithms, we have extended their applicability to a diverse range of problems spanning multiple fields, such as real-time object tracking, network management, and routing. Our approach involves expanding the database by introducing an additional dimension, similar to error-correction codes, which provides supplementary information, including the category of the search data points. We have applied this method to develop an algorithm for tracking moving particles, but its potential reaches far beyond this specific problem.}
	\cventry{08/2023-12/2023}{Krylov complexity in non-local systems}{}{}{}{\textbf{PI:} Dr. Aranya Bhattacharya (Institute of Physics, Jagiellonian University) \\
		\textbf{Team/Collaborators:} Pingal Pratyush Nath (Centre for High Energy Physics, Indian Institute of Science) \\ \textbf{Summary:} Motivated by recent works in spin systems with nonlocal interactions, this study investigates operator growth using the Lanczos algorithm in various versions of the Ising model. We find that the non-locality results in a faster scrambling of the operator to all sites. The corresponding Krylov complexities  still carry the distinguishability between integrable and chaotic theories in a suppressed way than the local Hamiltonian, which is a result of the faster scrambling for nonlocal Hamiltonian at early times. We investigate behavior of level statistics and spectral form factor as a probe of quantum chaos to study the integrability breaking due to non-local interactions.} 
	
	
	
	\cventry{11/2023-Present}{Spready Complexity of Random Unitary Circuits}{}{}{}{
		\textbf{Team/Collaborators:} Dr. Aranya Bhattacharya (Institute of Physics, Jagiellonian University), Pingal Pratyush Nath (Centre for High Energy Physics, Indian Institute of Science) \\ \textbf{Summary:} In ongoing work, I devised the formulation for evaluating the spread complexity of random unitary circuits. Using this formulation, we studied the complexity in random-haar unitary circuits as well as monitored RUCs.} 
	
		
	\cventry{12/2023-Present}{Information scrambling in Time Crystals}{}{}{}{
		\textbf{Team/Collaborators:} Fernando Iemini (Universidade Federal Fluminense), Pingal Pratyush Nath (Centre for High Energy Physics, Indian Institute of Science) \\ \textbf{Summary:} In ongoing work, We are studying information scrambling in Time crystals} 
	
	
	%\subsection{Activity in brief}
	%\cvitem{Publications}{Author of 5 publications}
	%\cvitem{Conferences \& Workshops}{}
	
	
	
	\section{Publications}
	\mysubsection{Peer-reviewed journals}
		\cventry{[1]}{\textnormal{A. Bhattacharya, P.P. Nath \& H. Sahu, Krylov complexity for non-local spin chains. \href{https://doi.org/10.1103/PhysRevD.109.066010}{Phys. Rev. D 109, 066010 (2024)} }}{}{}{}{\textcolor{blue}{All authors contributed equally to this work.}}
	
	\cventry{[2]}{\textnormal{A. Bhattacharya, H. Sahu, A. Zahed, and K. Sen, Complexity for one-dimensional Discrete Time Quantum Walk Circuits. \href{https://doi.org/10.1103/PhysRevA.109.022223}{Phys. Rev. A 109, 022223 (2024)}}}{}{}{}{}
	
	\cventry{[3]}{\textnormal{H. Sahu \& K. Sen, Quantum-walk search in motion.  \href{https://doi.org/10.1038/s41598-024-51709-0}{Scientific Reports 14, 2815 (2024)} }}{}{}{}{}
	
	
	\cventry{[4]}{\textnormal{H. Sahu \& C.M. Chandrashekar, Open system approach to Neutrino oscillations in a quantum walk framework. \href{https://doi.org/10.1007/s11128-023-04222-8}{Quantum Information Processing 23, 7 (2024)} }}{}{}{}{}
	
	\cventry{[5]}{\textnormal{A. Bhattacharya, P. Nandy, P.P. Nath \& H. Sahu, On Krylov complexity in open systems: an approach via bi-Lanczos algorithm. \href{https://link.springer.com/article/10.1007/JHEP12(2023)066}{Journal of High Energy Physics 2023, 66 (2023)} }}{}{}{}{\textcolor{blue}{All authors contributed equally to this work.}}
	
	
	\cventry{[6]}{\textnormal{A. Bhattacharya, P. Nandy, P.P. Nath \& H. Sahu, H. Operator growth and Krylov construction in dissipative open quantum systems. \href{https://doi.org/10.1007/JHEP12(2022)081}{Journal of High Energy Physics 2022, 81 (2022)}}}{}{}{}{\textcolor{blue}{All authors contributed equally to this work.}}
	
	

	\mysubsection{Pre-prints under review}

	\cventry{[1]}{\textnormal{A. Bhattacharya, P.P. Nath \& H. Sahu, Speed limits to the growth of Krylov complexity in open quantum systems, (2024). \href{https://arxiv.org/abs/2403.03584}{arXiv:2403.03584 [quant-ph]}}}{}{}{}{\textcolor{blue}{All authors contributed equally to this work.} \textcolor{darkgray}{In Physics Review D (Letter)}}


	\cventry{[2]}{\textnormal{K.V. Sharma, H. Sahu \& S. Mukerjee, Quantum chaos in $\mathcal{PT}$-symmetric Quantum Kicked Rotor, (2023). \href{https://doi.org/10.48550/arXiv.2401.07215}{arXiv:2401.07215 [quant-ph]}}}{}{}{}{}
	
	
	
	\mysubsection{In-preparation}
	

	\cventry{[1]}{\textnormal{H. Sahu, A. Bhattacharya, and P.P. Nath, Spready Complexity in Random Unitary Circuits}}{}{}{}{\textcolor{darkgray}{The order of the author and article title may vary in bibliographic citations.}}
	
	
	\mysubsection{Bibliometric parameters}
	\cvitem{Indices}{h-index 2 total citations 116 (May 2024), iNSPIRE-HEP}
	\cvitem{}{h-index 2 total citations 86 (May 2024), Google-Scholar}
	
	
	
	\section{Conferences, Seminars, and Schools}
	
	
	
	\mysubsection{Talks}
	\cventry{2023}{\textnormal{Quantum Information Scrambling in non-local systems}}{}{}{}{CHEP In-House Symposium, Centre for High Energy Physics, Indian Institute of Science, Bangalore, India 18-19 November 2023 }
	
	
	
	\mysubsection{Posters}
	
	\cventry{2023}{\textnormal{Simulating Neutrino Oscillations Using Quantum-walk }}{}{}{}{\href{https://www.hri.res.in/~confqic/qipa23/index.html}{Quantum Information Processing and Applications}, Harish-Chandra Research Institute, Prayagraj, India 04-10 December 2023.}
	
	
	\cventry{2023}{\textnormal{Quantum Information Scrambling in Dissipative Open Quantum Systems}}{}{}{}{\href{https://sites.google.com/iitpkd.ac.in/etqt23/}{Emerging Topics in Quantum Technology}, Indian Institute of Technology, Palakkad, India 02-04 November 2023.}
	
	\cventry{2023}{\textnormal{Operator Complexity in Open Quantum System}}{}{}{}{\href{https://www.icts.res.in/program/COMQUI23}{Condensed Matter meets Quantum Information}, International Centre for Theoretical Sciences (ICTS), Bengaluru, India 25 Sep-06 Oct 2023.}
	
	\cventry{2023}{\textnormal{Neutrino oscillations in discrete-time quantum walk framework}}{}{}{}{\href{https://prlstudentchapter.org.in/}{Student Conference in Optics and Photonics}, Physical Research Laboratory, Ahmedabad, India 27-29 September 2023.}
	
	\cventry{2023}{\textnormal{Exploring Operator Growth and Krylov Complexity in Dissipative Open Quantum Systems}}{}{}{}{\href{https://events.perimeterinstitute.ca/event/43/}{It from Qubit}, Perimeter Institute for Theoretical Physics, Waterloo, Ontario, Canada 31 July-4 August 2023 (Online)}
	
	
	\mysubsection{Attended Conferences}
	\cventry{2023}{\textnormal{\href{https://www.hri.res.in/~confqic/qipa23/index.html}{Quantum Information Processing and Applications}}}{}{}{}{Harish-Chandra Research Institute, Prayagraj, India 04-10 December 2023.}
	
	
	\cventry{2023}{\textnormal{\href{https://sites.google.com/iitpkd.ac.in/etqt23/}{Emerging Topics in Quantum Technology}}}{}{}{}{Indian Institute of Technology, Palakkad, India 02-04 November 2023.}
	
	\cventry{2023}{\textnormal{\href{https://prlstudentchapter.org.in/}{Student Conference in Optics and Photonics}}}{}{}{}{Physical Research Laboratory, Ahmedabad, India 27-29 September 2023.}
	
	\cventry{2023}{\textnormal{\href{https://www.photonics2023.net/}{Photonics 2023}}}{}{}{}{Indian Institute of Science, India 05-08 July 2023.}
	
	
	
	\mysubsection{Virtually Attended Conferences}
	\cventry{2023}{\textnormal{\href{https://events.perimeterinstitute.ca/event/43/}{It from Qubit}}}{}{}{}{Perimeter Institute for Theoretical Physics, Waterloo, Ontario, Canada 31 July-4 August 2023.}
	
	\cventry{2023}{\textnormal{\href{https://events.perimeterinstitute.ca/event/36/}{Machine Learning for Quantum Many-Body Systems }}}{}{}{}{Perimeter Institute for Theoretical Physics, Waterloo, Ontario, Canada 12-17 June 2023.}
	
	\cventry{2023}{\textnormal{\href{https://events.perimeterinstitute.ca/event/40/}{Quantum Simulators of Fundamental Physics }}}{}{}{}{Perimeter Institute for Theoretical Physics, Waterloo, Ontario, Canada 05-10 June 2023.}
	
	
	\mysubsection{Attended Schools}
	
	\cventry{2023}{\textnormal{\href{https://www.icts.res.in/program/COMQUI23}{Condensed Matter meets Quantum Information}}}{}{}{}{International Centre for Theoretical Sciences (ICTS), Bengaluru, India 25 Sep-06 Oct 2023}
	
	\cventry{2021}{\textnormal{\href{https://www.iucaa.in/en/education/summer-winter-programmes/introductory-summer-school-in-astronomy-and-astrophysics}{Introductory Summer School in Astronomy and Astrophysics}}}{}{}{}{Inter-University Centre for Astronomy and Astrophysics, India 10 May-11 June 2021.}
	
	
	
	
	\mysubsection{Seminars and Panels}
	
	\cventry{2023}{\textnormal{Climate Change and Disaster Risk Reduction : Making Sustainability a Way of Life}}{}{}{}{\href{https://y20india.in/}{Y20 Panel Discussion}, Indian Institute of Science, Bangalore, India 12 August 2023.}
	
	
	%\cvitem{2023} {Climate Change and Disaster Risk Reduction : Making Sustainability a Way of Life }
	%\cvitem{}{\href{https://y20india.in/}{Y20} Panel Discussion, \href{https://iisc.ac.in/}{Indian Institute of Science}, Bangalore, India 12 August 2023}
	
	\section{Teaching Activity}
	
	\mysubsection{Teaching Assistant}
	\cventry{Spring 2023}{\textnormal{UP 204 - Intermediate Thermal Physics}}{}{}{}{$\bullet$ Undergraduate Course Grader $\bullet$ Organized exams, graded exam sheets, and provided students with detailed feedback.   }
	%\cvitemwithcomment{2023}{\textnormal{UP 204 - Intermediate Thermal Physics}}{Course Grader}
	
	
	\section{Honors, Awards \& Scholarships}
	
\cventry{2023}{\textnormal{Semi-Finalist, \href{https://www.rhodeshouse.ox.ac.uk/scholarships/applications/india/}{Rhodes Scholarship}}} {}{}{}{$\bullet$ A semi-finalist for the prestigious Rhodes Scholarship, representing the STEM category. $\bullet$ Acknowledged for exceptional academic and leadership qualities during the Rhodes Scholarship application process }

\cventry{2021-2024}{\textnormal{IISC MS scholarship}}{}{}{}{$\bullet$ Beneficiary of the academic and financial provisions provided to Integrated PhD scholars at IISc Bangalore.}

\cventry{2018-2021}{\textnormal{\href{https://online-inspire.gov.in/}{INSPIRE Scholarship Awardee (SHE Program)}}}{}{}{}{$\bullet$ Awarded the prestigious INSPIRE Scholarship for Higher Education, a selective grant awarded to top $1\%$ performers in XII standard supported by the Department of Science and Technology, Government of India }
\cventry{2017}{\textnormal{\href{https://en.wikipedia.org/wiki/Free_laptop_distribution_scheme_of_the_Uttar_Pradesh_government}{State Government Academic Excellence Award}}}{}{}{}{$\bullet$ Acknowledged by the State Government for academic achievements in Class X}
	
	
	\section{Other skills}
	\mysubsection{Computer skills}
	\cvitem{OS}{Windows, Linux, HPC}
	\cvitem{Languages}{Python, Processing3, JavaScript, CSS, HTML}
	\cvitem{Software}{Mathematica, \LaTeX, Matlab, Microsoft office, Origin, $\ldots$}
	\cvitem{Libraries}{Numpy, Scipy, Qiskit, QuSpin, QuTip, Sympy, Open Fermion, joblib, p5.js, $\ldots$}
	
	\mysubsection{Linguistic skills}
	\cvitem{Hindi}{Mother tongue}
	\cvitem{English}{Fluent : TOEFL iBT Score - 99/120 (L:28-R:25-W:25-S:21)}
	
	
	\section{Volunteering}
	\cventry{2020-Present}{\textnormal{Contributor  on \href{https://stackexchange.com/users/16277143/young-kindaichi}{Physics Stack Exchange}}}{}{}{}{$\bullet$ Top $2\%$ overall $\bullet$ $11$K+ Reputation $\bullet$  $\sim 284$K people reached}
	
	\cventry{2023-Present}{\textnormal{IISC Nature Club Coordinator}}{}{}{}{$\bullet$ Raising awareness towards the need to protect environment $\bullet$ Organize walks in the forest and treks }
	
	\cventry{2023 \& 2024}{\textnormal{IISC Annual Open Day}}{}{}{}{$\bullet$ On 4th March 2023 $\bullet$ Presented experiments and counselled youths for a career in science. }
	
	\cvitem{2023} {Waste Collector in GOA Monsoon Trekking 2023}
	\cventry{2023}{\textnormal{Behavioral Experiment Test Subject Volunteer : Vision Lab IISC}}{}{}{}{}
	\cvitem{2023} {EEG Experiment Test Subject Volunteer : MILE Lab IISC}
	\cvitem{2023} {Functional MRI Experiment Test Subject Volunteer : Centre for Neuroscience IISC}
	
	\section{Extracurricular}
	
	\cventry{2023}{\textnormal{National Monsoon Trekking cum Training Expedition- Goa, 2023}}{}{}{}{Youth Hostels Association of India Goa State Brach, July-August 2023.}
	
	%\cvitem{2023}{National Monsoon Trekking cum Training Expedition- Goa, 2023}
	%\cvitem{}{Youth Hostels Association of India Goa State Brach, July-August 2023}
	
	%
	%\bibliography{pre_prints}                        % 'publications' is the name of a BibTeX file
	%\nocite{*}
	
	\section{References}
	
	\cventry{Prof.}{CM Chandrashekar}{\textnormal{Adjunt Faculty at Instrumentation and Applied Physics, Indian Institute of Sciences, Bengaluru, India}}{}{}{Email: \href{mailto:chandracm@iisc.ac.in}{chandracm@iisc.ac.in}}
	
	\cventry{Prof.}{Subroto Mukerjee}{\textnormal{Associate Professor, Indian Institute of Science}}{Bengaluru, India}{}{Email: \href{mailto:smukerjee@iisc.ac.in}{smukerjee@iisc.ac.in}}
	
	
	\cventry{Dr.}{Aranya Bhattacharya}{\textnormal{Postdoc at Institute of Physics, Jagiellonian University}}{Krakow, Poland}{}{Email: \href{mailto:aranya.bhattacharya@uj.edu.pl}{aranya.bhattacharya@uj.edu.pl}}
	
	\cventry{Prof.}{Sumilan Banerjee}{\textnormal{Associate Professor, Indian Institute of Science}}{Bengaluru, India}{}{Email: \href{mailto:sumilan@iisc.ac.in}{sumilan@iisc.ac.in}}
	
	
	
	\cventry{Dr.}{Kallol Sen}{\textnormal{Postdoc at ICTP-SAIFR, Sao Paulo}}{Brazil}{}{Email: \href{mailto:kallolmax@gmail.com}{kallolmax@gmail.com}}
	
	
	
	\section{Github Directories}
	\cvitem{\href{https://github.com/Kmax-art/q-complexity}{q-complexity}}{The directory contains code files related to work done in finding the quantum circuit complexity of quantum walk.}
	\cvitem{\href{https://github.com/Kmax-art/q-search}{q-search}}{The directory contains code files related to work done in quantum-walk search in motion.}
	
	
	\thispagestyle{plain}
	% Publications from a BibTeX file without multibib
	%  for numerical labels: \renewcommand{\bibliographyitemlabel}{\@biblabel{\arabic{enumiv}}}% CONSIDER MERGING WITH PREAMBLE PART
	%  to redefine the heading string ("Publications"): \renewcommand{\refname}{Articles}
	%\nocite{*}
	%\bibliographystyle{plain}
	%\bibliography{publications}                       % 'publications' is the name of a BibTeX file
	
	%
	%\clearpage
	%%-----       letter       ---------------------------------------------------------
	%% recipient data
	%\recipient{Professor}{OIST, Onna, Okinawa, Japan}
	%\date{\today}
	%\opening{Dear Professor,}
	%\closing{Yours faithfully,}
	%\enclosure[Attached]{curriculum vit\ae{}}          % use an optional argument to use a string other than "Enclosure", or redefine \enclname
	%\makelettertitle
	%
	%
	%I’ll pursue my research internship in quantum computation and information due to its interdisciplinary approach as it brings together ideas from information theory, computer science, and quantum physics, and its wide applicability from understanding fundamental physics to applications in cryptography, communication, and many others. There’s also a wide possibility of close collaboration between experiments, theory, and industry. Pursuing my research internship in physics at OIST will provide me with the rigorous training needed to develop my theoretical skills and conduct independent research. It would further add to my research journey toward Ph.D. in the same field. 
	%
	%
	%Over the past several years, I have been fortunate to have opportunities to conduct challenging projects in physics, through which I developed my interest in quantum computing. "But imagination alone is not enough because the reality of nature is far more wondrous than anything we can imagine." This quote from the TV series "Cosmos: A Spacetime Odyssey" resonated deeply, igniting a burning desire to learn more about the cosmos and contribute to our understanding. Inspired by the series, which introduced me to a vast array of natural phenomena, the history of civilization, and the individuals who expanded our knowledge of the world, I became determined to join the ranks of those unravelling the mysteries of nature. This passion, coupled with my status as an Inspire scholar, led me to pursue a bachelor’s degree in physics.
	%
	%During my undergraduate years, Quantum physics captivated me the most due to its contrasting nature with classical physics and its potential for unifying various branches of the field. Lectures by Balakrishnan provided me with invaluable mathematical rigor, while authors like R. Shankar, Steven H. Simon, and Purcell revealed the simplicity at the core of this captivating subject. I also took part in discussion in platform such as Physics stack exchange which gave me strong ground of basic concepts. I decided to pursue carrier in research and joined Indian Institute of Science as master’s student. 
	%
	%During the summer of 2022, I did three projects in various fields ranging from many-body system to quantum computation to high-energy physics. Some of these projects followed in my later semester projects. Here, I’ll describe few of these which added to my interest in quantum computation and information.
	%
	%As my first project, I worked with a Postdoc, Aranya, assigned to me as a guide by Prof. Aninda Sinha on quantum information scrambling (QIS), which can be thought of as mixing up qubits in way that they become entangled and intertwined with each other, creating a complex mess. The quantity called Krylov complexity quantifies how quantum information scrambles. The project focuses on extension of current formulation to study quantum information scrambling in dissipative quantum systems where information can flow to environment. In this project, my role was to independently study the system, thereby confirming the results obtained by collaborators. The results were later reported in JHEP as equally contributed author. The project was numerically intense, which made me accustomed to coding (language such as python). I also gained familiarity with various libraries such as Numpy, Scipy, Quspin, etc. 
	%
	%In the coming months, we explored Krylov complexity in various settings. For example, we considered QIS in PT-symmetric systems which can be thought of as system that interact with environment in a way that there’s a balance between the energy exchange. We are keeping this idea back of our mind, waiting for right idea to pop up. The study of PT-symmetric system come handy in my other projects as I discuss later.
	%
	%In early 2023, as a continuation to our previous work, we showed that Krylov complexity follows a universal behaviour in open quantum systems. In this project, I did numerical calculation of Krylov complexity for several models. I also did analytical calculation in support of numerical observation. These analytics require solving wave-functions that follow certain recurrence relation among themselves. I took a continuum limit over index of wave-function and constructed a partial differential equation. This PDE can be solved analytically in certain case. This taught me that numerical calculation are a good playground to verify your physical intuition in cases where the analytical calculation is complex. This work is reported and in review process (the last referee report is in positive). This works also added to my skills in working with others. Furthermore, since, the work is mainly with student at Ph.D. and Postdoc level, I able to immerse myself more in discussions. 
	%
	%
	%As my second project, I worked with Prof. C M Chandrashekar on quantum simulation based on the quantum walks. It started as self-structured survey of which I was in charge after which I settled to a problem of simulating neutrino oscillation. Neutrinos are nearly massless particles that are produced in various processes like nuclear reactions in the Sun and come in three different ‘flavors’: electron, muon, and tau. Neutrino oscillation happens because don’t stay in just one flavor. As they move through space, they can change from one flavor to another. I started to reproduce results from earlier literature, which took time. This is partially because most of the time literature do not provide details of worked out results, and sometime even the parameter values are missing. Therefore, I have refigured out these which in turn gave me a clear understanding of the work. During the literature survey, I always used to look for what can be improved in the approach which came handy in this paper. I realized that if you want to simulate neutrino dynamics with quantum walk the number of qubits required increase linearly as time. The linearly increasing qubit space comes from position space of particle. To overcome this problem, I propose a formulation that eliminate the requirement of position space by reconciling the effect of position space through introduction of additional operators. The work later reported and submitted for review. The work added to my understanding of simulating fundamental physics with quantum simulators, and furthermore, gained insight into independent research.
	%
	%As my master’s thesis, I decided to pursue quantum computation \& information in framework of quantum-walks due to its interdisciplinary approach that appeals to me. I again embarked into a self-structured literature survey and looked wide range of topics such as quantum memories for quantum repeaters, a device essential for exchange of quantum information over long distances. I’m trying to exploit recurrence theorem in quantum walk to devise quantum memory. According to recurrence theorem, the quantum state of a system repeat itself after certain time. However, previous studies have shown that theorem doesn’t hold in generic quantum walks. I devised a quantum walk formulation that revive the recurrence property which involve adding a periodic time dependence in system parameter. Currently, I’m looking into analytics of this formulation to provide expression for recurrence time. 
	%
	%Apart from this, I'm also looking into the quantum percolation model, which is used to study the behaviour of quantum particles, such as electrons or photons, in a disordered medium. In previous work, quantum simulation based on the quantum walk of such a model has been proposed. I'm trying to extend this to study transport in PT- symmetric disorder systems. We expect this to give a robust PT- symmetric phase that can support an Anderson localization transition, offering a rich phase diagram due to the interplay between disorder and PT- symmetry.
	%
	%Due to my wide collaboration interest, I joined Postdoc Kallol in a project that explores the interconnection between quantum complexity and circuit complexity. In Nielsen's geometric formulation, the circuit complexity of a unitary operator $U$ is the length of the minimal geodesic on the unitary ground joining the identity to $U$. The relation between this circuit complexity and circuit depth in actual quantum circuits is still not explored. In the coming months, we explored this problem by considering an example of quantum walk. We did explicit calculations of both the complexities and showed that both the complexities grow linear over time. I did explicit calculation of circuit complexity of quantum circuit as well as numerical investigation of neilsen complexity. We reported these results, which are currently under review. 
	%
	%In the realm of quantum computing, the quantum walk search algorithm is designed for locating fixed marked nodes within a graph. However, when multiple marked nodes exist, this conventional search method lacks the capacity to reveal their order. I proposed modification to search algorithm that overcome this problem. Leveraging this concept, we tackle the problem of tracking a moving particle in both space and time. Our algorithm efficiently searches for the trajectory of the mobile particle and is supported by a proposed quantum circuit. This concept holds promise for a range of applications, from real-time object tracking to network management and routing. The project taught me a lot about quantum algorithms, and quantum circuit construction for these. At the end, I also used to using libraries such as Qiskit. We are preparing an abstract for submission and a manuscript for publication.
	%
	%I am enthusiastic about the opportunity to pursue searches of new physics at OIST. OIST is at the frontier of both theoretical and experimental work in this field, and I am drawn to its strong collaborative culture. I am interested in working with Prof. Kae Nemoto, Prof. David Elkouss and Prof. Thomas Busch. After careful inspection about work of these faculties, I am confident I would thrive under their mentorship and expertise. I believe I have the tools to conduct successful research as a member of these theoretical as well as experimental teams. With my skills, my interdisciplinary background, and my ability to learn independently, I will be able to provide new ideas as a productive group member.
	%
	%My previous experiences have allowed me to develop as a researcher: independent and able to work collaboratively towards a common goal. I look forward to further learning, challenging myself, and developing new skills, and I am thrilled to have the opportunity to continue my interdisciplinary research and academic journey at OIST.
	%
	%
	%
	%
	%
	%
	%
	%
	%\makeletterclosing
	
	%\clearpage\end{CJK*}                              % if you are typesetting your resume in Chinese using CJK; the \clearpage is required for fancyhdr to work correctly with CJK, though it kills the page numbering by making \lastpage undefined
	\end{document}
	%%%
	%%%
	%%%%% end of file `template.tex'.
